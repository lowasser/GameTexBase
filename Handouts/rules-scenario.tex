\documentclass[sheet]{PP}

\usepackage{graphicx}
\graphicspath{ {./images/} }
\usepackage{xcolor}
\usepackage{hyperref}
\usepackage{multicol}
\usepackage{ltablex}
\usepackage{tabularx}
\usepackage{indentfirst}
\renewcommand{\tabularxcolumn}[1]{m{#1}}
\setlength{\columnsep}{1cm}
%
%%% document-wide tweaks
%\interlinepenalty10000
%\setstretch{1}
%\def\mytype{Rules and Scenario}
%\lfoot{}\rfoot{}
%
\begin{document}
\centering{\title{Pixie Perfect Rules Document}}

\section*{Interrupting someone:}
Actions cannot generally be interrupted. A character does a thing, and it happens. The only exception is ``rituals'' (see below). Rituals are a single activity that requires multiple steps. A character can interrupt a ritual by moving close enough that you could reach out and touch the other player. Point at them and say ``I stop you''. No touching required. \textit{(Fun fact: If you know what a ZoC (zone of control) is, this is HALF that distance.)}

\section*{Magic:}
Magic in this game is represented as abilities and greensheets. Expect your abilities to be themed to your character and their Court. The list of abilities you have are not the full extent of magic available to your character, but a subset of the ones we judge most likely to be useful during the game, without breaking the game balance so that the game isn’t fun for anyone else but you.

Some magical abilities will have a number of times they can be used in game, targeting or timing requirements, and/or consume a unit of \textbf{Pixie Dust} to use the ability, or to complete an action item on a greensheet. \textbf{Pixie Dust} is a specific, consumable item that Magic and Maker pixies are the most likely to have on hand.

\subsection*{Rituals:}
Some types of magic take place as rituals. These are usually described in greensheets and involve several steps to complete. You must always answer truthfully if someone asks if you are performing a ritual (it may not be obvious to players, but it is always obvious to characters). Rituals are always interruptible (see above). Some rituals will require you to start over if you are interrupted. Others are just paused. Check your ritual for specifics.

\section*{Combat:}
Violence is quite uncommon among pixies, but sometimes you may need to restrain someone or something, take something away from someone, or in the most extreme cases, knock something unconscious (e.g. a rampaging squirrel).

Combat is a straightforward matter of comparing how many people support a course of action vs those who oppose it. If you encounter a situation that cannot be resolved through IC discussion in which the involved characters agree on a course of action, you may attempt to force your course of action by invoking combat.

You can only instigate combat with characters who you could reach out and touch. 
\begin{itemize}
	\item To begin a combat, raise your hand in the air and call “combat” loudly and clearly (but you don’t need to shout). 
	\item Declare your intention (e.g.: I want to take that from you, I want to block this door that is currently clear, etc.) Your opposition must declare their intention as well (e.g.: I want to keep this item, I want to walk through this door, etc.)
	\item Determine together which side is trying to maintain the status quo. This side is considered the ``defenders.'' (e.g. The person holding the item wishes to continue holding onto it, the person who wants to walk through the currently clear door - blocking the door would change the status quo.)
	\item Other players within 1 ZoC must then choose to support your action, support your opposition, or stay neutral. The target of your action must oppose you, otherwise combat is not necessary. If no one opposes you, you automatically win.
	\item Once everyone within 1 ZoC has made a decision, count up how many people support each side. The side with the higher total wins. Ties go to the side upholding the status quo (“the defender”. E.g. the one currently holding the object you wish to take, the one currently moving through a space whom you are trying to rush in to stop, the one not currently restrained who does not want to become restrained.)
\end{itemize}

Anyone outside of 1 ZoC of either you or your target may not participate in the combat. Further, \textbf{you may not initiate another combat about the same thing for 2 minutes}. (e.g. you cannot try to grab the same object from another character repeatedly, you cannot try to force your way past someone constantly, or chase someone around trying to restrain them.) This represents your character having been overwhelmed by the opposition and needing to regroup before trying again. \textbf{However, another character can initiate the same or a similar action immediately, and you can add your support to their action.}

What exactly is the status quo is very situational. If someone is already blocking a door, the status quo is that they succeed in preventing your passage. If someone is NOT currently blocking a door, but wishes to move into position to stop someone already moving through a door, the status quo is that the door is not blocked, and the character can freely walk through it. \textbf{The status quo can also change between combats.} If the “attackers” win a combat to try to block someone from going somewhere, their action has succeeded, and “the door is blocked by this character” is now the status quo, and they would be the “defenders” in a subsequent combat.

There are \textbf{never} more than \textbf{2} sides to a combat, and the line between targeting one character vs a whole group is all in the wording of the declaration (e.g.: “I want to restrain you” vs. “I want to restrain anyone who tries to stop me.”) Other characters should feel completely empowered to stay out of encounters, or even support the defenders if the statement you make is too extreme for their taste (e.g.: someone might be willing to support you trying to restrain someone, but not trying to knock them out, or willing to help you restrain 1 specific person, but not “anybody”.)

\section*{Stealth:} There is no stealth mechanic for combat in this game.

\section*{Health State and Death:} You cannot instigate combat to kill another character. \textbf{The most harm you can do is to knock someone out.} Pixies will find such violence excessive in all but the most dire of circumstances, and you will likely find yourself without the allies necessary to accomplish it without very good reason.

A knocked out character will wake up in 60 seconds. A restrained character will escape their restraints in 2 minutes, unless someone's full attention is on keeping them restrained.

\end{document}