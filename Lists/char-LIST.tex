%%%%%
%%
%% This file sets up the Char, PC, NPC, and Name datatypes and creates
%% macros for each.  These are for characters in the game.  Here you
%% use the fields in Char to assign elements to each character.
%%
%%
%%
%% \MYname (and the player name) is parsed by \parsename, the command
%% provided by LaTeX/parsename.sty.  See that file and
%% Extras/README-namemappings for ways to take advantage of this.
%%
%%
%%
%% \MYCharpronouns is set to either \hehim, \sheher, \neuter, or \ambiguous, as
%% correct for the character.  \mfn{<masculine>}{<feminine>}{<neuter>}
%% will produce the correct form based on the current value of \MYCharpronouns
%% (\ambiguous will lead to <masculine>/<feminine>).  \mfn should only
%% be used within the scope of a Char macro.  \mf{<masc>}{<fem>} works
%% just like \mfn with the <neut> arg left blank.
%%
%% \pronoun{<command>}{<masc>}{<fem>}{<neut>} makes <command> a
%% wrapper around \mfn.  It is used to create a list of
%% gender-sensitive macros, mostly pronouns.  For example, given
%% \pronoun{\They}{He}{She}{It}, \cJamesBond{\They} will produce He.
%%
%%
%%
%% \badgetrue and \badgefalse toggle whether or not a Char macro will
%% produce a namebadge.
%%
%% \statstrue and \statsfalse will toggle the statcard.
%%
%% \skillstrue and \skillsfalse will toggle the skill list.  The skill
%% list prints both skills and stats (even if \statsfalse is set).
%%
%% \sheettrue and \sheetfalse will toggle the character sheet.
%%
%% \listtrue and \listfalse toggle whether the Char macro can display
%% in the playerlist.
%%
%% \labeltrue and \labelfalse toggle the box label.
%%
%%
%%
%% Some of the Char datatype's setup is in LaTeX/gametex.sty, to keep
%% this file short.
%%
%%%%%



%%%%%
%% If a field is declared by \F, it must be given a value by \s inside
%% \NEW, even if it's blank.  If you want it to be optional, declare
%% it with \FD<field> {<default>} here.
%%
%% Use \newstat to create stats (below, in \PRESETS{Char}).  The
%% <default> value is used unless the given Char macro sets the field.
%% For example:
%%
%%   \newstat\MYhp	{Hit Points}{HP}{5}
%%
%% would give character a Hit Points stat, abbreviated HP, referenced
%% as the \MYhp field, that defaults to 5.
\PRESETS{Char}{
  \FD\MYfirst{}
	\FD\MYlast{}
	\FD\MYdesc	{} %% badge description

	\newstat\MYcr	{Combat Rating}{CR}{2} %% for DarkWater-style combat
	\newstat\MYcourt	{Pixie Court}{PC}{} % Courts: Structure, F&F, Elements, M&M

  \FD\MYCharpronouns	{} %% \hehim, \sheher, \neuter, \ambiguous
  \FD\MYPlaypronouns	{} %% \hehim, \sheher, \neuter, \ambiguous
	\FD\MYrole	{} %% playerlist role
  \FD\MYgroupstr{} %% playerlist groupstring
  \FD\MYfile	{} %% character sheet filename (including .tex)
  \FS\MYtext	{\ifx\MYfile\empty\else%
		  \getextractenvs{document}{\chars/\MYfile}%
		\fi}
  \badgetrue\statstrue\skillstrue\sheettrue\listtrue\labeltrue
  }

\POSTSETS{Char}{
  \resolvestats
  }


%%%%%
%% pronouns and similar gender-based macros
%%
%% \hehim, \sheher, \neuter, \ambiguous
%% \mfn{<masculine>}{<feminine>}{<neuter>}
%% \mf{<masculine>}{<feminine>}
%% \pronoun{<command>}{<masculine>}{<feminine>}{<neuter>}
\def\hehim{0}\def\sheher{1}\def\theythem{2}\def\zezir{3}\def\faefaer{4}
\newcommand{\mfnzf}[5]{\ifcase\MYCharpronouns#1\or#2\or#3\or#4\else#5\fi}
\newcommand{\mfnz}[4]{\mfnz{#1}{#2}{#3}{#4}{#3}}
\newcommand{\mfn}[3]{\mfnz{#1}{#2}{#3}{#3}}
\newcommand{\mf}[2]{\mfn{#1}{#2}{}}
\newcommand{\pronounzf}[6]{\def#1{\mfnzf{#2}{#3}{#4}{#5}{#6}}}
\newcommand{\pronounz}[5]{\def#1{\mfnz{#2}{#3}{#4}{#5}}}
\newcommand{\pronoun}[4]{\def#1{\mfn{#2}{#3}{#4}}}

\pronounzf{\they}		{he}{she}{they}{ze}{fae}
\pronounzf{\They}		{He}{She}{They}{Ze}{Fae}
\pronounzf{\them}		{him}{her}{them}{zir}{faer}
\pronounzf{\their}	{his}{her}{their}{zir}{faer}
\pronounzf{\Their}	{His}{Her}{Their}{Zir}{Faer}
\pronounzf{\theirs}	{his}{hers}{theirs}{zirs}{faers}
\pronounzf{\themself}	{himself}{herself}{themself}{zirself}{faerself}
\pronounzf{\Themself}	{Himself}{Herself}{Themself}{Zirself}{Faerself}
\newcommand{\theyare}{\they \are}
\newcommand{\Theyare}{\They \are}

\pronoun{\verbs}	{s}{s}{}
\pronoun{\verbes}	{es}{es}{}
\pronoun{\verby}	{ies}{ies}{y}
\pronounz{\are}	{is}{is}{are}{is}
\pronounz{\re}	{'s}{'s}{'re}{'s}
\newcommand{\does}{do\verbes}
\pronounz{\have}	{has}{has}{have}{has}
\pronounz{\were}	{was}{was}{were}{was}

\pronoun{\spouse}	{husband}{wife}{spouse}
\pronoun{\Spouse}	{Husband}{Wife}{Spouse}
\pronoun{\child}	{son}{daughter}{child}
\pronoun{\Child}	{Son}{Daughter}{Child}
\pronoun{\kid}		{boy}{girl}{child}
\pronoun{\Kid}		{Boy}{Girl}{Child}
\pronoun{\stray}	{urchin}{waif}{stray}
\pronoun{\Stray}	{Urchin}{Waif}{Stray}
\pronoun{\pal}		{guy}{gal}{pal}
\pronoun{\Pal}		{Guy}{Gal}{Pal}
\pronoun{\sibling}	{brother}{sister}{sibling}
\pronoun{\Sibling}	{Brother}{Sister}{Sibling}
\pronoun{\parent}	{father}{mother}{parent}
\pronoun{\Parent}	{Father}{Mother}{Parent}
\pronoun{\auncle}	{uncle}{aunt}{auncle}
\pronoun{\Auncle}	{Uncle}{Aunt}{Auncle}
\pronoun{\nibling}	{nephew}{niece}{nibling}
\pronoun{\Nibling}	{Nephew}{Niece}{Nibling}
%% \def\aunt{\uncle}
%% \def\Aunt{\Uncle}
\pronounzf{\person}	{man}{woman}{person}{person}{fae}
\pronoun{\Person}	{Man}{Woman}{Person}
\pronoun{\sex}		{male}{female}{nonbinary}
\pronoun{\Sex}		{Male}{Female}{Nonbinary}
\pronoun{\God}		{God}{Goddess}{Deity}
\pronoun{\Deity}		{God}{Goddess}{Deity}
\pronoun{\Majesty}		{King}{Queen}{Majesty}
\pronoun{\Heir}		{Prince}{Princess}{Heir}
\pronoun{\heir}		{Prince}{Princess}{Heir}
\pronoun{\cleric}		{Priest}{Priestess}{Cleric}
\pronoun{\clergy}		{Priest}{Priestess}{Cleric}
\pronoun{\partner}	{boyfriend}{girlfriend}{partner}

%%%%%
%% PC is a subtype of Char, for regular PCs.
\DECLARESUBTYPE{PC}{Char}
\PRESETS{PC}{\sd\MYgroupstr{pc}}


%%%%%
%% NPC is a subtype of Char.
\DECLARESUBTYPE{NPC}{Char}
\PRESETS{NPC}{\badgefalse\sd\MYgroupstr{npc}}


%%%%%
%% Name is a subtype of NPC.
%% For an in-text name.  By default, produces no packet material.
\DECLARESUBTYPE{Name}{Char}
\PRESETS{Name}{
  \badgefalse\statsfalse\skillsfalse\sheetfalse\listfalse\labelfalse
  \sd\MYgroupstr{name}
  }


%%%%%%%%%%%%%%%%%%%%%%%%%%%%%%%%%%%%%%%%%%%%%%%%%%%%%%%%%%%%%%%%%%

%% don't use \cTest as a copy-and-paste template to populate your
%% character list.  Use something simpler, like
%%
%%   \NEW{PC}{\cBlah}{
%%     \s\MYname	{}
%%     \s\MYfile	{}
%%     }
%%
%\NEW{PC}{\cTest}{
%  \s\MYname	{Test Character}
%  \s\MYfile	{README.tex}
%  \s\MYnumber	{00000}
%  \s\MYdesc	{a test}
%  \s\MYplayer	{Test Player}
%  \s\MYemail	{test@test.test}
%  \s\MYaddress	{Test, rm 000}
%  \s\MYphone	{x0-0000}
%  \s\MYblues	{\bTest{}}
%  \s\MYgreens	{\gTest{}\nGreenTest{}}
%  \s\MYabils	{\aTest{}
%		\aTestCombat{}
%		}
%  \s\MYitems	{\iTest{}\nTest{}}
%  \s\MYwhites	{\wTest{}}
%  \s\MYcash	{\cash{Dollar}{261}}
%  }


%\NEW{PC}{\cTest}{
  %\s\MYname	{Test Character}
  %\s\MYfile	{README.tex}
  %\s\MYnumber	{00000}
  %\s\MYdesc	{a test}
  %\MYsex	{\male}
  %\s\MYplayer	{Test Player}
  %
  %\s\MYblues	{\bTest{}}
  %\s\MYgreens	{\gTest{}\nGreenTest{}}
  %\s\MYabils	{\aTest{}\aTestCombat{}}
  %\s\MYitems	{\iTest{}\nTest{}}
  %\s\MYwhites	{\wTest{}}
  %\s\MYcash	{\cash{Dollar}{261}}
  %
  %\s\MYcr{2}
%}

%%%%%%%%%%%%% The PCS %%%%%%%%%%%%%%%
