%%%%%
%%
%% This file sets up the Char, PC, NPC, and Name datatypes and creates
%% macros for each.  These are for characters in the game.  Here you
%% use the fields in Char to assign elements to each character.
%%
%%
%%
%% \MYname (and the player name) is parsed by \parsename, the command
%% provided by LaTeX/parsename.sty.  See that file and
%% Extras/README-namemappings for ways to take advantage of this.
%%
%% \badgetrue and \badgefalse toggle whether or not a Char macro will
%% produce a namebadge.
%%
%% \statstrue and \statsfalse will toggle the statcard.
%%
%% \skillstrue and \skillsfalse will toggle the skill list.  The skill
%% list prints both skills and stats (even if \statsfalse is set).
%%
%% \sheettrue and \sheetfalse will toggle the character sheet.
%%
%% \listtrue and \listfalse toggle whether the Char macro can display
%% in the playerlist.
%%
%% \labeltrue and \labelfalse toggle the box label.
%%
%%
%%
%% Some of the Char datatype's setup is in LaTeX/gametex.sty, to keep
%% this file short.
%%
%%%%%

%%%%%
%% Gender
%%
%% The term 'gender' specifically in GameTeX is used to refer to the
%% pronoun and gendered language used to actually refer to a character.
%% To add a new neopronoun, add a new element to this list
%% with the appropriate next number.
\def\mascgender{0}
\def\femmegender{1}
\def\theygender{2}
\def\zegender{3}
\def\faegender{4}
\def\avoidpronounsgender{5}
\def\eygender{6}
\def\itgender{7}

%% Here, we define 'pronoun sets,' which have two components:
%% which gender is used to always refer to that character in a sentence,
%% and the text used when we introduce that character's pronouns.
%% This does not cover all combinations, but it's easy to add more.
\newcommand{\pronounset}[3]{
	\newcommand{#1}[1][]{
		\ifx#1\empty
			\s\MYgender{#2}
			\s\MYcharpronouns{#3}
		\else
			\rs\MYgender{#2}
			\rs\MYcharpronouns{#3}
		\fi
	}
}

\pronounset{\hehim}{\mascgender}{he/him}
\pronounset{\hethey}{\mascgender}{he/they}
\pronounset{\theyhe}{\theygender}{they/he}
\pronounset{\heshe}{\mascgender}{he/she}
\pronounset{\shehe}{\femmegender}{she/he}
\pronounset{\sheher}{\femmegender}{she/her}
\pronounset{\shethey}{\femmegender}{she/they}
\pronounset{\theyshe}{\theygender}{they/she}
\pronounset{\theythem}{\theygender}{they/them}
\pronounset{\anypronouns}{\theygender}{any pronouns}
\pronounset{\genderfluid}{\theygender}{any pronouns}
\pronounset{\zezir}{\zegender}{ze/zir}
\pronounset{\zethey}{\zegender}{ze/they}
\pronounset{\theyze}{\theygender}{they/ze}
\pronounset{\faefaer}{\faegender}{fae/faer}
\pronounset{\eyem}{\eygender}{ey/em}
\pronounset{\itits}{\itgender}{it/its}

\pronounset{\avoidpronouns}{\avoidpronounsgender}{avoid pronouns}
	% When referring to a character with this gender, we use \informal to e.g. use 'John' instead of 'John Smith'.

%%%%%
%% Here, we define words and conjugations that are sensitive to a character's gender.
%% If you add a new pronoun set, you will need to update all of these accordingly.

% Switch on gender, falling back to they/them if there's an error.
\newcommand{\gendered}[8]{\ifcase\MYgender#1\or#2\or#3\or#4\or#5\or#6\or#7\or\8\else#3\fi}
% define a new command that switches based on gender
\newcommand{\foreachgender}[9]{\def#1{\gendered{#2}{#3}{#4}{#5}{#6}{#7}{#8}{#9}}}
% define one version for he/she/ze/it, one version for they/fae/ey
\newcommand{\singularorplural}[3]{\foreachgender{#1}{#2}{#2}{#3}{#2}{#3}{#2}{#3}{#2}}
% define one version for masc, one for femme, one for nonbinary
\newcommand{\mfornb}[4]{\foreachgender{#1}{#2}{#3}{#4}{#4}{#4}{#4}{#4}{#4}}

%% Words that need a different version for every individual gender.
\foreachgender{\they}		{he}{she}{they}{ze}{fae}{\informal}{ey}{it}
\foreachgender{\They}		{He}{She}{They}{Ze}{Fae}{\informal}{Ey}{It}
\foreachgender{\them}		{him}{her}{them}{zir}{faer}{\informal}{em}{it}
\foreachgender{\their}	{his}{her}{their}{zir}{faer}{\informal's}{eir}{its}
\foreachgender{\Their}	{His}{Her}{Their}{Zir}{Faer}{\informal's}{eir}{Its}
\foreachgender{\theirs}	{his}{hers}{theirs}{zirs}{faers}{\informal's}{eirs}{its}
\foreachgender{\themself}	{himself}{herself}{themself}{zirself}{faerself}{\informal}{emself}{itself}
\foreachgender{\Themself}	{Himself}{Herself}{Themself}{Zirself}{Faerself}{\informal}{emself}{Itself}

%% Words that simply care about pluralization or not.
\singularorplural{\verbs}		{s}{}  		% he believes, they believe
\singularorplural{\verbes}	{es}{} 		% he does, they do
\singularorplural{\verby}		{ies}{y} 	% he worries, they worry
\singularorplural{\are}			{is}{are} % he is, they are
\singularorplural{\re}			{'s}{'re} % he's, they're
\def\does{do\verbes}
\singularorplural{\have}		{has}{have} % he has, they have
\singularorplural{\were}		{was}{were} % he was, they were

%% Nouns that simply care about masculine, feminine, or 'other.'
%% Depending on your game, you may find that you need to add to this list -- feel free!

% family words
\mfornb{\spouse}	{husband}{wife}{spouse}
\mfornb{\Spouse}	{Husband}{Wife}{Spouse}
\mfornb{\offspring}	{son}{daughter}{offspring}
\mfornb{\Offspring}	{Son}{Daughter}{Offspring}
\mfornb{\child}		{boy}{girl}{child}
\mfornb{\Child}		{Boy}{Girl}{Child}
\mfornb{\sibling}	{brother}{sister}{sibling}
\mfornb{\Sibling}	{Brother}{Sister}{Sibling}
\mfornb{\sib} {bro}{sis}{sib}
\mfornb{\Sib} {Bro}{Sis}{Sib}
\mfornb{\parent}	{father}{mother}{parent}
\def\grandparent{grand\parent}
\def\Grandparent{Grand\parent}
\mfornb{\Parent}	{Father}{Mother}{Parent}
\mfornb{\auncle}	{uncle}{aunt}{auncle}
\mfornb{\Auncle}	{Uncle}{Aunt}{Auncle}
\mfornb{\nibling}	{nephew}{niece}{nibling}
\mfornb{\Nibling}	{Nephew}{Niece}{Nibling}
\mfornb{\grandchild} {grandson}{granddaughter}{grandchild}
\mfornb{\Grandchild} {Grandson}{Granddaughter}{Grandchild}

\mfornb{\partner}	{boyfriend}{girlfriend}{partner}
\mfornb{\betrothed} {fianc\'e}{fianc\'ee}{betrothed}

%% Noble titles.  Opinions on the best nonbinary versions vary, but this should be a good start.
\mfornb{\Heir}		{Prince}{Princess}{Heir}
\mfornb{\heir}		{prince}{princess}{heir}
\mfornb{\Monarch}		{King}{Queen}{Monarch}
\mfornb{\monarch}		{king}{queen}{monarch}
\mfornb{\Emperor}		{Emperor}{Empress}{Emprex}
\mfornb{\emperor}		{emperor}{empress}{emprex}
\mfornb{\Duke} 		{Duke}{Duchess}{Duke}
\mfornb{\Count}		{Count}{Countess}{Count}
\mfornb{\Baron}		{Baron}{Baroness}{Baron}

%% Other titles of respect.
\mfornb{\Cleric}		{Priest}{Priestess}{Cleric}
% To do things like Mx. Smith, it's probably better to use name prefixes than gender-sensitive macros; see LaTeX/parsename.sty.

\mfornb{\stray}	{urchin}{waif}{stray}
\mfornb{\Stray}	{Urchin}{Waif}{Stray}
\mfornb{\pal}		{guy}{gal}{pal}
\mfornb{\Pal}		{Guy}{Gal}{Pal}
\mfornb{\person}	{man}{woman}{person}
\mfornb{\Person}	{Man}{Woman}{Person}
\mfornb{\sex}		{male}{female}{nonbinary}
\mfornb{\Sex}		{Male}{Female}{Nonbinary}
\mfornb{\Deity}		{God}{Goddess}{Deity}
\def\God{\Deity}


%%%%%
%% If a field is declared by \F, it must be given a value by \s inside
%% \NEW, even if it's blank.  If you want it to be optional, declare
%% it with \FD<field> {<default>} here.
%%
%%%% Character stats
%% Use \newstat to create stats (below, in \PRESETS{Char}).  The
%% <default> value is used unless the given Char macro sets the field.
%% For example:
%%
%%   \newstat\MYhp	{Hit Points}{HP}{5}
%%
%% would give character a Hit Points stat, abbreviated HP, referenced
%% as the \MYhp field, that defaults to 5.
\PRESETS{Char}{
	\FD\MYdesc	{}   %% badge description
  \F\MYgender	{}	 %% must be set by a command like \hehim etc.
	\FD\MYcharpronouns{}   %% must be set by a command like \hehim etc.
	\FD\MYplayerpronouns{}   %% DO NOT use \theythem, write ``they/them'' or whatever
	\FD\MYrole	{} %% playerlist role
  \FD\MYgroupstr{} %% playerlist groupstring
  \FD\MYfile	{} %% character sheet filename (including .tex)
  \FS\MYtext	{\ifx\MYfile\empty\else%
		  \getextractenvs{document}{\chars/\MYfile}%
		\fi}
		
	%\newstat\MYhp	{Hit Points}{HP}{5}
  \badgetrue\statstrue\skillstrue\sheettrue\listtrue\labeltrue
  }

\POSTSETS{Char}{
  \resolvestats
  }
	
%%%%% Defines pronounful contact and intro mappings.
\intromap{\optionalparen{\full}{\MYcharpronouns}}
\contactmap{\optionalparen{\full}{\MYcharpronouns}}

%%%%%
%% PC is a subtype of Char, for regular PCs.
\DECLARESUBTYPE{PC}{Char}
\PRESETS{PC}{\sd\MYgroupstr{pc}}


%%%%%
%% NPC is a subtype of Char.
\DECLARESUBTYPE{NPC}{Char}
\PRESETS{NPC}{\badgefalse\sd\MYgroupstr{npc}}


%%%%%
%% Name is a subtype of NPC.
%% For an in-text name.  By default, produces no packet material.
\DECLARESUBTYPE{Name}{Char}
\PRESETS{Name}{
  \badgefalse\statsfalse\skillsfalse\sheetfalse\listfalse\labelfalse
  \sd\MYgroupstr{name}
  }


%%%%%%%%%%%%%%%%%%%%%%%%%%%%%%%%%%%%%%%%%%%%%%%%%%%%%%%%%%%%%%%%%%

%% \cTest demonstrates all the possible options for a character, but
%% you may not even need all of these for every game, so feel free
%% to skip any options you don't need.
%%
%%   \NEW{PC}{\cBlah}{
%%     \s\MYname	{}
%%     \s\MYfile	{}
%%     }
%%
%\NEW{PC}{\cTest}{
%  \s\MYname	{Test Character}
%  \s\MYfile	{MyCharSheet.tex}		% a file in the Charsheets directory
%  \s\MYnumber	{00000}						% badge number
%  \s\MYdesc	{a test}						% a few words of description that can fit
%																	% visibly on a badge
%  \s\MYplayer	{Test Player}			% player name
%  \hethey												% configures pronouns
%  \s\MYblues	{\bTest{}}					% assigned bluesheets
%  \s\MYgreens	{\gTest{}\nGreenTest{}}			% assigned greensheets
%  \s\MYabils	{\aTest{}						% assigned abilities -- demonstrates you can split lines in these lists
%		\aTestCombat{}
%		}
%  \s\MYitems	{\multi{5}{\iTest{}}\nTest{}}	% assigned items, including notebooks
%  \s\MYwhites	{\wTest{}}				% assigned whitesheets
%  % \s\MYcr{2}										% if you create stats in your game, this is how you set them for each character
%  }

%%%%%%%%%%%%% The PCs %%%%%%%%%%%%%%%

\NEW{PC}{\cTest}{
  \s\MYname	{John Doe}
	\s\MYfile {README.tex} % to fill in with a file name from the Charsheets directory
  \s\MYdesc	{a nondescript person}  % short description to fit on a badge
  \hethey												% configures pronouns
	\s\MYnumber{1234}
	\s\MYplayer {Coolest Roleplayer}
	\s\MYplayerpronouns{they/them}
	\s\MYblues {\bTest{}}
	\s\MYgreens {\gTest{}}
	\s\MYwhites {}
	\s\MYmems {\mTest{}}
	\s\MYabils {\aTest{}}
	\s\MYitems {}
}

%%%%%%%%%%%%% The NPCs %%%%%%%%%%%%%%%

\NEW{NPC}{\cTestNPC}{
	\s\MYname {Chef \pre Someone}
	\theythem
}
\NEW{NPC}{\cSomeGuy}{
	\s\MYname {Mr.\pre Someone Dude}
	\hehim
}