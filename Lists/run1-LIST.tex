%%%%%
%%
%% GameTeX will input the file run\gamerun-LIST.tex.  So if \gamerun
%% is set to 1, run1-LIST.tex will be used, if 2, run2-LIST.tex, etc.
%% The file is not required; if it does not exist, no error will
%% occur.  run1-LIST.tex is likely unnecessary.
%%
%% You can use this file for run-dependent modifications (usually
%% player information in Char datatypes).
%%
%% \updatemacro{<datatype macro>}{<new updates>}
%% \updateplayer{<Char macro>}{<new player>}
%% \updateplayeremail{<Char macro>}{<new player>}{<new email>}
%% \updatePNG{<Char macro>}{<new player>}{<new char name>}{<new gender>}
%%
%% \updatemacro will extend the contents of a datatype macro.  Use \rs
%% to change field values:
%%
%%   \updatemacro{\cTest}{
%%     \rs\MYplayer	{New Player}
%%     \sheher{overwrite}
%%   }
%%
%% \updateplayer is a shortcut for changing just the player name for a
%% character:
%%
%%   \updateplayer{\cTest}{New Player}
%%
%% \updatePNG is a shortcut for changing, well:
%%
%%   \updatePNG{\cTest}{New Playername}{they/them}{New Charactername}{\sheher}
%% where the last two options can be left blank if desired.
%% Note that the third argument is the new set of player pronouns.
%%
%%%%%
